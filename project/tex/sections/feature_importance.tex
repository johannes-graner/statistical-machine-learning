\documentclass[../../project.tex]{subfiles}
\graphicspath{{\subfix{images/}}}

\begin{document}
	In order to determine which of the features 'Year', 'Gross' or 'Number of female actors' is most important, we fitted models excluding one or more of the features. Using a logistic regression model with LASSO (L1) regularization, we found that the model performance in terms of prediction accuracy was completely unaffected by removing either or both of the features 'Year' and 'Gross'. On the other hand, any model that excluded the variable 'Proportion of words female' (which contains all information about how much male and female actors speak), had its performance reduced drastically. As seen in Table \ref{tab:logreg_table_90}, the model including all features had an accuracy of 0.9. This dropped to 0.8 when removing the 'Proportion of words female'. Comparing this to the null accuracy of 0.756, we conclude that the 'Proportion of words female' is a very important feature in the model. Interestingly, this 'Proportion of words female' seems very important for all models, but especially so for QDA; this suggests that QDA may depend more heavily on this input for predictions. The same results hold for both QDA and k-NN as well, the models are completely unaffected by removing either 'Year' or 'Gross' (or both), while suffering 0.08 accuracy loss for k-NN and 0.12 loss for QDA. Further, not only accuracy is affected but AUC is affected in the same way.% ,showing that 'Year' and 'Gross' are more or less redundant features in the model, while 'Proportion of words female' is incredibly important for predicting whether the lead is male or female.
\end{document}