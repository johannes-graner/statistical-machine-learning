\documentclass[../../project.tex]{subfiles}
\graphicspath{{\subfix{images/}}}

\begin{document}
    We answer the following questions assuming that the given data set is representative of Hollywood movies in general.
\begin{enumerate}
    \item \textbf{Do men or women dominate speaking roles in Hollywood?} In the given data we see that in $75.6\%$ of movies there is a male in the lead. Further, the mean of 'Proportion of words female' is $34,6\%$ indicating that Hollywood movies consist of almost $65\%$ male speaking.
    
    \item \textbf{Has gender balance in speaking roles changed over time (i.e. years)?} None of the models considered suffered performance loss from removing the input variable 'Year'. However, the correlation in the given data set between 'Year' and all of 'Numbers of words female', 'Number of female actors' and 'Mean Age female' is positive and not trivially small (0.032, 0.134 and 0.170 respectively). We conclude that this indicates a slight change in the gender balance over time but that the model does not necessarily use this relationship to make predictions.
   
    \item \textbf{Do films in which men do more speaking make a lot more money than films in which women speak more?} None of the models considered suffered performance loss from removing the input variable 'Gross' and the QDA model even improved a bit indicating that 'Gross' may be a bad input variable for this model. This might be because over the years, the change in 'Gross' for men and women may have changed at different rates and because of inflation; this makes 'Gross' a potentially bad input variable for future predictions.
\end{enumerate}
	
All of the models fitted in this project focus on prediction instead of inference, hence it is not possible to use them to answer the above questions directly.
    
	In order to determine which of the features 'Year', 'Gross' or 'Number of female actors' is most important, we fitted models excluding one or more of the features. Using a logistic regression model with LASSO (L1) regularization, we found that the model performance in terms of prediction accuracy was completely unaffected by removing either or both of the features 'Year' and 'Gross'. On the other hand, any model that excluded the variable 'Proportion of words female' (which contains all information about how much male and female actors speak), had its performance reduced drastically. As seen in Table \ref{tab:logreg_table_90}, the model including all features had an accuracy of 0.9. This dropped to 0.8 when removing the 'Proportion of words female'. Comparing this to the null accuracy of 0.756, we conclude that the 'Proportion of words female' is a very important feature in the model. Interestingly, this 'Proportion of words female' seems very important for all models, but especially so for QDA; this suggests that QDA may depend more heavily on this input for predictions. The same results hold for both QDA and k-NN as well, the models are completely unaffected by removing either 'Year' or 'Gross' (or both), while suffering 0.08 accuracy loss for k-NN and 0.12 loss for QDA. Further, not only accuracy is affected but AUC is affected in the same way.% ,showing that 'Year' and 'Gross' are more or less redundant features in the model, while 'Proportion of words female' is incredibly important for predicting whether the lead is male or female.
\end{document}