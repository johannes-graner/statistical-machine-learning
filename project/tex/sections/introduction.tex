\documentclass[../../project.tex]{subfiles}
\graphicspath{{\subfix{images/}}}

\begin{document}
	In a movie, there is almost always a main character that the rest of the film is in some way centered around. In some films the main character is not very nice, but most of the time, the audience is expected to sympathize with this character or even look up to, and be inspired by them. This is especially true in children's movies. Since it is often easier to be inspired by someone you can relate to, it is important to make sure that the distribution of main characters in movies at least approximates the distribution of the audience. If the main character is always a man, that makes it more difficult than necessary for girls and women to find someone to be inspired by and vice versa.
	
	This paper investigates how the gender of the lead actor (who plays the main character), can be predicted using metrics such as the year when the movie was made and the age of the lead actor. Being able to make such predictions could give an insight into how the movie industry works with respect to the gender of the lead actor. This could provide clues about what areas to investigate further to understand why those connections exist and ultimately, make sure that everyone has a chance to be inspired by someone they can relate to.
\end{document}